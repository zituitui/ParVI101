\documentclass{article}

\usepackage[UTF8]{ctex}

\usepackage{graphicx} 
\usepackage{subfigure}

\usepackage{booktabs}

\usepackage{amsfonts}
\usepackage{amsmath}
\usepackage{amssymb}
\usepackage{amsthm}
\usepackage{bm}
\usepackage{bbm}
\usepackage{cases}
\usepackage{footmisc}
\usepackage{geometry}
\usepackage[colorlinks]{hyperref}

\geometry{a4paper, scale=0.7}

\newtheorem{definition}{定义}[section]
\newtheorem{theorem}{定理}[section]
\newtheorem{lemma}{引理}[section]
\newtheorem{remark}{注}[section]

\usepackage[ruled,linesnumbered]{algorithm2e}
\renewcommand{\algorithmcfname}{算法}
\author{WangFangyikang}

\allowdisplaybreaks

\begin{document}
    \title{关于Par VI的若干问题}
    \maketitle
    \tableofcontents

    \newpage
    \section{Computational Complexity}
我们定义$L,(P,r,\gamma)$是确定一个MDP所需的bit-size,假定算术运算+,-,$\times$,$\div$都使用单位时间(unit time)。我们希望找到一个算法,它能找到在$L,(P,r,\gamma)$、$|mathcal{S}|$、$|mathcal{A}|$的多项式时间内找到最优策略。
如果一个算法能在$|S|$、$|A|$的多项式时间(与$L,(P,r,\gamma)$无关)内找到最优策略,则称该算法为strongly polynomial。
\subsection{Value Iteration}
\begin{lemma}[收缩性]
    对任意两个向量$Q$,$Q'\in \mathbb{R}^{|mathcal{S}||mathcal{A}|}$,$\Vert \tau Q-\tau Q' \Vert_\infty\leq\gamma\Vert Q-Q' \Vert_\infty$。
    这说明了$\tau$是一个收缩映射。
 \end{lemma}
\begin{proof}
    首先证明对于所有的$s\in \mathcal{S},|V_Q(s)-V_{Q'(s)}|\leq \underset{a\in \mathcal{A}}{max}|Q(s,a)-Q'(s,a)|$。\\
    不妨设$V_Q(s)>V_{Q'}(s)$(另一半是对称的),记$a^*=\underset{a}{argmax}Q(s,a)$。\\
    $|V_Q(s)-V_{Q'(s)}|=\underset{a\in\mathcal{A}}{max}Q(s,a)-\underset{a\in\mathcal{A}}{max}Q'(s,a)\leq Q(s,a^*)-Q'(s,a^*)\leq\underset{a\in\mathcal{A}}{max}|Q(s,a)-Q'(s,a)|$
    \begin{equation}
        \begin{aligned}
            \Vert \tau Q-\tau Q' \Vert_\infty &= \Vert r+\gamma PV_Q-r-\gamma PV_{Q'} \Vert_\infty\\
            &= \gamma \Vert P(V_Q-V_{Q'}) \Vert_\infty\\
            &\leq \gamma \Vert V_Q-V_{Q'} \Vert_\infty\\
            &= \gamma \underset{s}{\max}|V_Q-V_{Q'}|\\
            &\leq \underset{s}{\max}\underset{a}{\max}|Q(s,a)-Q'(s,a)|\\
            &= \gamma\Vert Q-Q' \Vert_\infty
        \end{aligned}
    \end{equation}
\end{proof}
由该引理可见,经历了k次迭代后,$\Vert Q_k-Q^* \Vert_\infty\leq\gamma^k\Vert Q_0-Q^*\Vert_\infty$,因此$\underset{k\to \infty}{\lim}Q_k-Q^*=0$,算法收敛。

\begin{lemma}[Q-Error Amplification)]
    对任意向量$Q\in \mathbb{R}^{|mathcal{S}||mathcal{A}|},V^{\pi_Q}\geq V^*-\frac{2\Vert Q-Q^*\Vert_\infty}{1-\gamma}\mathbbm{1}$
    \\其中$\mathbbm{1}$为全1向量。
 \end{lemma}
\begin{proof}
    对于固定的$s$以及$a=\pi_Q(s),(\pi_Q(s)=\underset{a\in \mathcal{A}}{argmax}Q(s,a))$
    \begin{equation}
        \begin{aligned}
            V^*(s)-V^{\pi_Q}(s)&=Q^*(s,\pi^*(s))-Q^{\pi_Q}(s,a)\\
            &=Q^*(s,\pi^*(s))-Q^*(s,a)+Q^*(s,a)-Q^(\pi_Q)(s,a)\\
            &=Q^*(s,\pi^*(s))-Q^*(s,a)+\gamma \mathbb{E}_{s'\sim P(\cdot |s,a)}[V^*(s')-V^{\pi_Q}(s')]\\
            &\leq Q^*(s,\pi^*(s))-Q(s,\pi^*(s))+Q(s,a)-Q^*(s,a)+\gamma \mathbb{E}_{s'\sim P(\cdot |s,a)}[V^*(s')-V^{\pi_Q}(s')]\\
            &\leq 2\Vert Q-Q^* \Vert_\infty+\gamma\Vert V^*-V^{\pi_Q}\Vert_\infty
        \end{aligned}
    \end{equation}
\end{proof}

\begin{theorem}[Q-value iteration convergence]
    设$Q^{(0)}=0,Q^{(k+1)}=\tau Q^{(k)},k=0,1,\ldots \pi^{(k)=pi_{Q^{(k)}}}$,当$k\geq \frac{\log \frac{2}{(1-\gamma )^2\epsilon}}{1-\gamma}$,\\
    $V^{\pi^{(k)}}\geq V^*-\epsilon \mathbbm{1}$,即k次迭代后$V^{\pi^{(k)}}$和$V^*$非常接近。
\end{theorem}
\begin{proof}
    $\Vert Q^{(k)}-Q^*\Vert_\infty=\Vert \tau ^kQ^{(0)}-\tau ^kQ^*\Vert_\infty \leq\gamma^k\Vert Q^{(0)}-Q^*\Vert_\infty=(1-()1-\gamma)^k\leq \frac{exp(-(1-\gamma)k)}{1-\gamma} $
\end{proof}

\subsection{Value Iteration}
策略迭代算法从任意一个策略$\pi_0$出发,并对k=0,1,2,$\ldots $重复接下来的两步:
1.策略评估。计算$Q^{\pi_k}$\\
2.策略提升。更新策略:$\pi_{k+1}=\pi_{Q^{\pi_k}}$,即$\pi_{k+1}(s)=\underset{a\in\mathcal{A}}{argmax}Q^{\pi_k}(s,a)$

\begin{lemma}
    1.$Q^{\pi_{k+1}}\geq \tau Q^{\pi_k}\geq Q^{\pi_k}$\\
    2.$\Vert Q^{\pi_{k+1}}-Q^* \Vert_\infty\leq\gamma\Vert Q^{\pi_k}-Q^* \Vert_\infty$
\end{lemma}
\begin{proof}
    首先证明$\tau Q^{\pi_k}\geq Q^{\pi_k}$。注意到策略迭代中的策略都是确定策略,䫅对于所有$k,s$,$V^{\pi_k}(s)=Q^{\pi_k}(s,\pi_k(s))$。\\
    \begin{equation}
        \begin{aligned}
            \tau Q^{\pi_k}(s,a)&=r(s,a)+\gamma \mathbb{E}_{s'\sim P(\cdot |s,a)}[\underset{a'}{\max}Q^{\pi_k}(s',a')]\\
            &\geq r(s,a)+\gamma\mathbb{E}_{s'\sim P(\cdot |s,a)}[Q^{\pi_k}(s',\pi_k(s'))]=Q^{\pi_k}(s,a)
        \end{aligned}
    \end{equation}
    \\再证明$Q^{\pi_{k+1}}\geq \tau Q^{\pi_k}$,这需要先证明$Q^{\pi_{k+1}}\geq Q^{\pi_k}$:\\
    $Q^{\pi_k}=r+\gamma P^{\pi_k}Q^{\pi_k}\leq r+\gamma P^{\pi_{k+1}}Q^{\pi_k}\leq \sum_{t=0}^{\infty}\gamma^t(P^{\pi_{k+1}})^tr=Q^{\pi_{k+1}}$\\
    第一个不等式是因为$\pi_{k+1}$是greedy policy,所以一定比$\pi_{k}$更好,第二个不等式由递归得到。\\因此
    \begin{equation}
        \begin{aligned}
            Q^{\pi_{k+1}}(s,a)&=r(s,a)+\gamma \mathbb{E}_{s'\sim P(\cdot |s,a)}[Q^{\pi_{k+1}}(s',\pi_{k+1}(s'))]\\
            &\geq r(s,a)+\gamma \mathbb{E}_{s'\sim P(\cdot |s,a)}[Q^{\pi_{k}}(s',\pi_{k+1}(s'))]\\
            &=r(s,a)+\gamma \mathbb{E}_{s'\sim P(\cdot |s,a)}[\underset{a'}{\max}Q^{\pi_k}(s',a')]=\tau Q^{\pi_k}(s,a)
        \end{aligned}
    \end{equation}
    (1)式证明完成。
    下面证明(2)式:\\
    $\Vert Q^*-Q^{\pi_{k+1}} \Vert_\infty\leq\Vert Q^*-\tau Q^{\pi_{k}} \Vert_\infty =\Vert \tau Q^*-\tau Q^{\pi_{k}} \Vert_\infty\leq\gamma\Vert Q^*-Q^{\pi_k} \Vert_\infty$\\
    证明完成。
\end{proof}

\begin{theorem}[Policy iteration convergence]
    设$Q^{\pi_0}=0,\pi_0$为任意初始策略。当$k\geq \frac{\log\frac{1}{(1-\gamma )\epsilon}}{1-\gamma} $,第k个策略有这样的performance bound:\\
    $Q^{\pi_k}\geq Q^*-\epsilon $
\end{theorem}

\begin{proof}
    \begin{equation}
        \begin{aligned}
            \Vert Q^*-Q^{\pi_k}\Vert_\infty&\leq\gamma\Vert Q^*-Q^{\pi_{k-1}}\Vert_\infty
            &\leq \gamma^k\Vert Q^*-Q^{\pi_0}\Vert_\infty
            &=\gamma^k\Vert Q^\Vert_\infty
            &=(1-(1-\gamma))^k\Vert Q^\Vert_\infty
            &\leq \frac{\exp (-(1-\gamma)k)}{1-\gamma} 
        \end{aligned}
    \end{equation}
\end{proof}


\subsection{Linear Programming Approach}
线性规划的方法可以在严格多项式时间内解决问题。\\
\subsubsection{原始问题}
最初的想法是求解\\
$\qquad\underset{V\in \mathbb{R}^{|\mathcal{S}|}}{\min}\qquad{\sum_s\mu (s)V(s)}$\\
subject to  $V(s)\geq\underset{a\in \mathcal{A}}{\max}[r(s,a)+\gamma\underset{s'}{\sum}{P(s'|s,a)V(s')}] \qquad \forall s\in\mathcal{S}$\\
但这不是LP问题。将其转化为LP问题,得到:\\
$\qquad\underset{V\in \mathbb{R}^{|\mathcal{S}|}}{\min}\qquad{\sum_s\mu (s)V(s)}$\\
subject to  $V(s)\geq r(s,a)+\gamma\underset{s'}{\sum}{P(s'|s,a)V(s')} \qquad \forall a\in\mathcal{A},s\in\mathcal{S} $\\
其中,$\mu (s)$是初始状态分布。如果$\mu$ has full support,那么该问题的唯一解就是$V^*(s)$。
\begin{proof}
    证明利用了$\tau$的单调性,即$V \geq V^{'}$时,有$\tau V \geq \tau V^{'}$。\\
    令$V'=\tau V$,则$\tau V\geq\tau V'=\tau^2V$。迭代得到:$V\geq\tau^\infty=V^*$。\\
    因此该约束条件下得到的解都是$V\geq V^*$的情况,由于目标函数是求$\qquad\underset{V\in \mathbb{R}^{|\mathcal{S}|}}{\min}\qquad{\sum_s\mu (s)V(s)}$,最终得到的解即$V=V^*$。
\end{proof}
\subsubsection{对偶问题}
对每一个LP都存在一个对偶问题,原问题的决策变量对应对偶问题的约束条件,原问题的约束条件对应对偶问题的决策变量。\\
对于固定的策略$\pi$,定义关于状态和动作的visitation measure:
\begin{equation}
    d^{\pi}_{s_0}(s,a)=(1-\gamma)\sum_{t=0}^{\infty}\gamma^tPr^\pi (s_t=s,a_t=a|s_0)
\end{equation}
其中$Pr^\pi (s_t=s,a_t=a|s_0)$是从状态$s_0$出发,经过策略$\pi$,到达$s_t=s,a_t=a$的概率。并记$d^{\pi}_{\mu}(s,a)=\mathbb{E}_{s_0\sim\mu}[d^{\pi}_{s_0}(s,a)]$\\
对于任意的状态$s$有:
\begin{equation}
    \underset{a}{\sum}d^{\pi}_{\mu}(s,a)=(1-\gamma)\mu(s)+\gamma\underset{s',a'}{\sum}P(s|s',a')d^{\pi}_{\mu}(s',a')
\end{equation}

\begin{proof}
    左边:\\
    \begin{equation}
        \begin{aligned}
            \underset{a}{\sum}d^{\pi}_{\mu}(s,a)&=d^{\pi}_{\mu}(s)\\
           &=\underset{s_0\sim \mu}{\mathbb{E}}[d^{\pi}_{s_0}(s)]\\
           &=\underset{s_0\sim \mu}{\mathbb{E}}[(1-\gamma)\sum_{t=0}^{\infty}\gamma^tPr^{\pi}(s_t=s|s_0)]
        \end{aligned}
    \end{equation}
    右边:
    \begin{equation}
        \begin{aligned}
            &\gamma\underset{s',a'}{\sum}P(s|s',a')d^{\pi}_{\mu}(s',a')+(1-\gamma)\mu(s)\\
            =&\gamma\underset{s',a'}{\sum}P(s|s',a')\underset{s_0\sim \mu}{\mathbb{E}}[(1-\gamma)\sum_{t=0}^{\infty}\gamma^tPr^{\pi}(s_t=s',a_t=a'|s_0)]+(1-\gamma)\mu(s)\\
            =&\underset{s_0\sim \mu}{\mathbb{E}}[(1-\gamma)\sum_{t=0}^{\infty}\sum_{s',a'}\gamma^{t+1}P(s|s',a')Pr^{\pi}(s_t=s',a_t=a'|s_0)]+(1-\gamma)\mu(s)\\
            =&\underset{s_0\sim \mu}{\mathbb{E}}[(1-\gamma)\sum_{t=0}^{\infty}\sum_{s',a'}\gamma^{t+1}Pr^{\pi}(s|s_0)]+(1-\gamma)\mu(s)\\
            =&\underset{s_0\sim \mu}{\mathbb{E}}[(1-\gamma)\sum_{t=0}^{\infty}\gamma^tPr^{\pi}(s_t=s|s_0)]\\
        \end{aligned}
    \end{equation}
    所以左边=右边。
\end{proof}
定义一个状态-动作多面体:\\
$\mathcal{K}_\mu :=\{d|d\geq 0\quad and \quad \sum_a d(s,a)=(1-\gamma)\mu (s)+\gamma\sum_{s',a'}P(s|s',a')d(s',a')\}$\\
可以看到,$\mathcal{K} _\mu$是所有可以的状态-动作分别的集合,即$d\in\mathcal{K} _\mu$当且仅当存在一个平稳的策略$\pi$使得$d^{\pi}_{\mu}=d$。\\
有了这些定义,对偶问题可以写成:\\
$\max\qquad\quad \frac{1}{1-\gamma}\sum_{s,a}d_\mu(s,a)r(s,a) $\\
subject to  $d\in \mathcal{K}_{\mu}$\\
目标函数可以看作是每个状态-动作对的密度乘上奖励,加权求和来求总奖励,求这个总奖励的最大值。解这个对偶LP求得一个解$d^*$,那么就可以求得最优策略:\\
$\pi^*(s,a)=\frac{d^*(s,a)}{\sum_{a'}d^*(s,a')}$\\
另一种求最优策略的方法是求$\underset{a}{argmax}\quad d^*(s,a')$。
    \section{Computational Complexity}
我们定义$L,(P,r,\gamma)$是确定一个MDP所需的bit-size,假定算术运算+,-,$\times$,$\div$都使用单位时间(unit time)。我们希望找到一个算法,它能找到在$L,(P,r,\gamma)$、$|mathcal{S}|$、$|mathcal{A}|$的多项式时间内找到最优策略。
如果一个算法能在$|S|$、$|A|$的多项式时间(与$L,(P,r,\gamma)$无关)内找到最优策略,则称该算法为strongly polynomial。
\subsection{Value Iteration}
\begin{lemma}[收缩性]
    对任意两个向量$Q$,$Q'\in \mathbb{R}^{|mathcal{S}||mathcal{A}|}$,$\Vert \tau Q-\tau Q' \Vert_\infty\leq\gamma\Vert Q-Q' \Vert_\infty$。
    这说明了$\tau$是一个收缩映射。
 \end{lemma}
\begin{proof}
    首先证明对于所有的$s\in \mathcal{S},|V_Q(s)-V_{Q'(s)}|\leq \underset{a\in \mathcal{A}}{max}|Q(s,a)-Q'(s,a)|$。\\
    不妨设$V_Q(s)>V_{Q'}(s)$(另一半是对称的),记$a^*=\underset{a}{argmax}Q(s,a)$。\\
    $|V_Q(s)-V_{Q'(s)}|=\underset{a\in\mathcal{A}}{max}Q(s,a)-\underset{a\in\mathcal{A}}{max}Q'(s,a)\leq Q(s,a^*)-Q'(s,a^*)\leq\underset{a\in\mathcal{A}}{max}|Q(s,a)-Q'(s,a)|$
    \begin{equation}
        \begin{aligned}
            \Vert \tau Q-\tau Q' \Vert_\infty &= \Vert r+\gamma PV_Q-r-\gamma PV_{Q'} \Vert_\infty\\
            &= \gamma \Vert P(V_Q-V_{Q'}) \Vert_\infty\\
            &\leq \gamma \Vert V_Q-V_{Q'} \Vert_\infty\\
            &= \gamma \underset{s}{\max}|V_Q-V_{Q'}|\\
            &\leq \underset{s}{\max}\underset{a}{\max}|Q(s,a)-Q'(s,a)|\\
            &= \gamma\Vert Q-Q' \Vert_\infty
        \end{aligned}
    \end{equation}
\end{proof}
由该引理可见,经历了k次迭代后,$\Vert Q_k-Q^* \Vert_\infty\leq\gamma^k\Vert Q_0-Q^*\Vert_\infty$,因此$\underset{k\to \infty}{\lim}Q_k-Q^*=0$,算法收敛。

\begin{lemma}[Q-Error Amplification)]
    对任意向量$Q\in \mathbb{R}^{|mathcal{S}||mathcal{A}|},V^{\pi_Q}\geq V^*-\frac{2\Vert Q-Q^*\Vert_\infty}{1-\gamma}\mathbbm{1}$
    \\其中$\mathbbm{1}$为全1向量。
 \end{lemma}
\begin{proof}
    对于固定的$s$以及$a=\pi_Q(s),(\pi_Q(s)=\underset{a\in \mathcal{A}}{argmax}Q(s,a))$
    \begin{equation}
        \begin{aligned}
            V^*(s)-V^{\pi_Q}(s)&=Q^*(s,\pi^*(s))-Q^{\pi_Q}(s,a)\\
            &=Q^*(s,\pi^*(s))-Q^*(s,a)+Q^*(s,a)-Q^(\pi_Q)(s,a)\\
            &=Q^*(s,\pi^*(s))-Q^*(s,a)+\gamma \mathbb{E}_{s'\sim P(\cdot |s,a)}[V^*(s')-V^{\pi_Q}(s')]\\
            &\leq Q^*(s,\pi^*(s))-Q(s,\pi^*(s))+Q(s,a)-Q^*(s,a)+\gamma \mathbb{E}_{s'\sim P(\cdot |s,a)}[V^*(s')-V^{\pi_Q}(s')]\\
            &\leq 2\Vert Q-Q^* \Vert_\infty+\gamma\Vert V^*-V^{\pi_Q}\Vert_\infty
        \end{aligned}
    \end{equation}
\end{proof}

\begin{theorem}[Q-value iteration convergence]
    设$Q^{(0)}=0,Q^{(k+1)}=\tau Q^{(k)},k=0,1,\ldots \pi^{(k)=pi_{Q^{(k)}}}$,当$k\geq \frac{\log \frac{2}{(1-\gamma )^2\epsilon}}{1-\gamma}$,\\
    $V^{\pi^{(k)}}\geq V^*-\epsilon \mathbbm{1}$,即k次迭代后$V^{\pi^{(k)}}$和$V^*$非常接近。
\end{theorem}
\begin{proof}
    $\Vert Q^{(k)}-Q^*\Vert_\infty=\Vert \tau ^kQ^{(0)}-\tau ^kQ^*\Vert_\infty \leq\gamma^k\Vert Q^{(0)}-Q^*\Vert_\infty=(1-()1-\gamma)^k\leq \frac{exp(-(1-\gamma)k)}{1-\gamma} $
\end{proof}

\subsection{Value Iteration}
策略迭代算法从任意一个策略$\pi_0$出发,并对k=0,1,2,$\ldots $重复接下来的两步:
1.策略评估。计算$Q^{\pi_k}$\\
2.策略提升。更新策略:$\pi_{k+1}=\pi_{Q^{\pi_k}}$,即$\pi_{k+1}(s)=\underset{a\in\mathcal{A}}{argmax}Q^{\pi_k}(s,a)$

\begin{lemma}
    1.$Q^{\pi_{k+1}}\geq \tau Q^{\pi_k}\geq Q^{\pi_k}$\\
    2.$\Vert Q^{\pi_{k+1}}-Q^* \Vert_\infty\leq\gamma\Vert Q^{\pi_k}-Q^* \Vert_\infty$
\end{lemma}
\begin{proof}
    首先证明$\tau Q^{\pi_k}\geq Q^{\pi_k}$。注意到策略迭代中的策略都是确定策略,䫅对于所有$k,s$,$V^{\pi_k}(s)=Q^{\pi_k}(s,\pi_k(s))$。\\
    \begin{equation}
        \begin{aligned}
            \tau Q^{\pi_k}(s,a)&=r(s,a)+\gamma \mathbb{E}_{s'\sim P(\cdot |s,a)}[\underset{a'}{\max}Q^{\pi_k}(s',a')]\\
            &\geq r(s,a)+\gamma\mathbb{E}_{s'\sim P(\cdot |s,a)}[Q^{\pi_k}(s',\pi_k(s'))]=Q^{\pi_k}(s,a)
        \end{aligned}
    \end{equation}
    \\再证明$Q^{\pi_{k+1}}\geq \tau Q^{\pi_k}$,这需要先证明$Q^{\pi_{k+1}}\geq Q^{\pi_k}$:\\
    $Q^{\pi_k}=r+\gamma P^{\pi_k}Q^{\pi_k}\leq r+\gamma P^{\pi_{k+1}}Q^{\pi_k}\leq \sum_{t=0}^{\infty}\gamma^t(P^{\pi_{k+1}})^tr=Q^{\pi_{k+1}}$\\
    第一个不等式是因为$\pi_{k+1}$是greedy policy,所以一定比$\pi_{k}$更好,第二个不等式由递归得到。\\因此
    \begin{equation}
        \begin{aligned}
            Q^{\pi_{k+1}}(s,a)&=r(s,a)+\gamma \mathbb{E}_{s'\sim P(\cdot |s,a)}[Q^{\pi_{k+1}}(s',\pi_{k+1}(s'))]\\
            &\geq r(s,a)+\gamma \mathbb{E}_{s'\sim P(\cdot |s,a)}[Q^{\pi_{k}}(s',\pi_{k+1}(s'))]\\
            &=r(s,a)+\gamma \mathbb{E}_{s'\sim P(\cdot |s,a)}[\underset{a'}{\max}Q^{\pi_k}(s',a')]=\tau Q^{\pi_k}(s,a)
        \end{aligned}
    \end{equation}
    (1)式证明完成。
    下面证明(2)式:\\
    $\Vert Q^*-Q^{\pi_{k+1}} \Vert_\infty\leq\Vert Q^*-\tau Q^{\pi_{k}} \Vert_\infty =\Vert \tau Q^*-\tau Q^{\pi_{k}} \Vert_\infty\leq\gamma\Vert Q^*-Q^{\pi_k} \Vert_\infty$\\
    证明完成。
\end{proof}

\begin{theorem}[Policy iteration convergence]
    设$Q^{\pi_0}=0,\pi_0$为任意初始策略。当$k\geq \frac{\log\frac{1}{(1-\gamma )\epsilon}}{1-\gamma} $,第k个策略有这样的performance bound:\\
    $Q^{\pi_k}\geq Q^*-\epsilon $
\end{theorem}

\begin{proof}
    \begin{equation}
        \begin{aligned}
            \Vert Q^*-Q^{\pi_k}\Vert_\infty&\leq\gamma\Vert Q^*-Q^{\pi_{k-1}}\Vert_\infty
            &\leq \gamma^k\Vert Q^*-Q^{\pi_0}\Vert_\infty
            &=\gamma^k\Vert Q^\Vert_\infty
            &=(1-(1-\gamma))^k\Vert Q^\Vert_\infty
            &\leq \frac{\exp (-(1-\gamma)k)}{1-\gamma} 
        \end{aligned}
    \end{equation}
\end{proof}


\subsection{Linear Programming Approach}
线性规划的方法可以在严格多项式时间内解决问题。\\
\subsubsection{原始问题}
最初的想法是求解\\
$\qquad\underset{V\in \mathbb{R}^{|\mathcal{S}|}}{\min}\qquad{\sum_s\mu (s)V(s)}$\\
subject to  $V(s)\geq\underset{a\in \mathcal{A}}{\max}[r(s,a)+\gamma\underset{s'}{\sum}{P(s'|s,a)V(s')}] \qquad \forall s\in\mathcal{S}$\\
但这不是LP问题。将其转化为LP问题,得到:\\
$\qquad\underset{V\in \mathbb{R}^{|\mathcal{S}|}}{\min}\qquad{\sum_s\mu (s)V(s)}$\\
subject to  $V(s)\geq r(s,a)+\gamma\underset{s'}{\sum}{P(s'|s,a)V(s')} \qquad \forall a\in\mathcal{A},s\in\mathcal{S} $\\
其中,$\mu (s)$是初始状态分布。如果$\mu$ has full support,那么该问题的唯一解就是$V^*(s)$。
\begin{proof}
    证明利用了$\tau$的单调性,即$V \geq V^{'}$时,有$\tau V \geq \tau V^{'}$。\\
    令$V'=\tau V$,则$\tau V\geq\tau V'=\tau^2V$。迭代得到:$V\geq\tau^\infty=V^*$。\\
    因此该约束条件下得到的解都是$V\geq V^*$的情况,由于目标函数是求$\qquad\underset{V\in \mathbb{R}^{|\mathcal{S}|}}{\min}\qquad{\sum_s\mu (s)V(s)}$,最终得到的解即$V=V^*$。
\end{proof}
\subsubsection{对偶问题}
对每一个LP都存在一个对偶问题,原问题的决策变量对应对偶问题的约束条件,原问题的约束条件对应对偶问题的决策变量。\\
对于固定的策略$\pi$,定义关于状态和动作的visitation measure:
\begin{equation}
    d^{\pi}_{s_0}(s,a)=(1-\gamma)\sum_{t=0}^{\infty}\gamma^tPr^\pi (s_t=s,a_t=a|s_0)
\end{equation}
其中$Pr^\pi (s_t=s,a_t=a|s_0)$是从状态$s_0$出发,经过策略$\pi$,到达$s_t=s,a_t=a$的概率。并记$d^{\pi}_{\mu}(s,a)=\mathbb{E}_{s_0\sim\mu}[d^{\pi}_{s_0}(s,a)]$\\
对于任意的状态$s$有:
\begin{equation}
    \underset{a}{\sum}d^{\pi}_{\mu}(s,a)=(1-\gamma)\mu(s)+\gamma\underset{s',a'}{\sum}P(s|s',a')d^{\pi}_{\mu}(s',a')
\end{equation}

\begin{proof}
    左边:\\
    \begin{equation}
        \begin{aligned}
            \underset{a}{\sum}d^{\pi}_{\mu}(s,a)&=d^{\pi}_{\mu}(s)\\
           &=\underset{s_0\sim \mu}{\mathbb{E}}[d^{\pi}_{s_0}(s)]\\
           &=\underset{s_0\sim \mu}{\mathbb{E}}[(1-\gamma)\sum_{t=0}^{\infty}\gamma^tPr^{\pi}(s_t=s|s_0)]
        \end{aligned}
    \end{equation}
    右边:
    \begin{equation}
        \begin{aligned}
            &\gamma\underset{s',a'}{\sum}P(s|s',a')d^{\pi}_{\mu}(s',a')+(1-\gamma)\mu(s)\\
            =&\gamma\underset{s',a'}{\sum}P(s|s',a')\underset{s_0\sim \mu}{\mathbb{E}}[(1-\gamma)\sum_{t=0}^{\infty}\gamma^tPr^{\pi}(s_t=s',a_t=a'|s_0)]+(1-\gamma)\mu(s)\\
            =&\underset{s_0\sim \mu}{\mathbb{E}}[(1-\gamma)\sum_{t=0}^{\infty}\sum_{s',a'}\gamma^{t+1}P(s|s',a')Pr^{\pi}(s_t=s',a_t=a'|s_0)]+(1-\gamma)\mu(s)\\
            =&\underset{s_0\sim \mu}{\mathbb{E}}[(1-\gamma)\sum_{t=0}^{\infty}\sum_{s',a'}\gamma^{t+1}Pr^{\pi}(s|s_0)]+(1-\gamma)\mu(s)\\
            =&\underset{s_0\sim \mu}{\mathbb{E}}[(1-\gamma)\sum_{t=0}^{\infty}\gamma^tPr^{\pi}(s_t=s|s_0)]\\
        \end{aligned}
    \end{equation}
    所以左边=右边。
\end{proof}
定义一个状态-动作多面体:\\
$\mathcal{K}_\mu :=\{d|d\geq 0\quad and \quad \sum_a d(s,a)=(1-\gamma)\mu (s)+\gamma\sum_{s',a'}P(s|s',a')d(s',a')\}$\\
可以看到,$\mathcal{K} _\mu$是所有可以的状态-动作分别的集合,即$d\in\mathcal{K} _\mu$当且仅当存在一个平稳的策略$\pi$使得$d^{\pi}_{\mu}=d$。\\
有了这些定义,对偶问题可以写成:\\
$\max\qquad\quad \frac{1}{1-\gamma}\sum_{s,a}d_\mu(s,a)r(s,a) $\\
subject to  $d\in \mathcal{K}_{\mu}$\\
目标函数可以看作是每个状态-动作对的密度乘上奖励,加权求和来求总奖励,求这个总奖励的最大值。解这个对偶LP求得一个解$d^*$,那么就可以求得最优策略:\\
$\pi^*(s,a)=\frac{d^*(s,a)}{\sum_{a'}d^*(s,a')}$\\
另一种求最优策略的方法是求$\underset{a}{argmax}\quad d^*(s,a')$。

\end{document}